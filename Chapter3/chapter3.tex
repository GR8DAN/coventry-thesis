%!TEX root = ../thesis.tex
%*******************************************************************************
%****************************** Third Chapter *********************************
%*******************************************************************************

\chapter{Examples of text}

\ifpdf
    \graphicspath{{Chapter3/Figs/Raster/}{Chapter3/Figs/PDF/}{Chapter3/Figs/}}
\else
    \graphicspath{{Chapter3/Figs/Vector/}{Chapter3/Figs/}}
\fi

\section{Lorem Ipsum}

Lorem ipsum dolor sit amet, consectetur adipiscing elit. Sed vitae laoreet lectus. Donec lacus quam, malesuada ut erat vel, consectetur eleifend tellus. Aliquam non feugiat lacus. Interdum et malesuada fames ac ante ipsum primis in faucibus. Quisque a dolor sit amet dui malesuada malesuada id ac metus. Phasellus posuere egestas mauris, sed porta arcu vulputate ut. Donec arcu erat, ultrices et nisl ut, ultricies facilisis urna. Quisque iaculis, lorem non maximus pretium, dui eros auctor quam, sed sodales libero felis vel orci. Aliquam neque nunc, elementum id accumsan eu, varius eu enim. Aliquam blandit ante et ligula tempor pharetra. Donec molestie porttitor commodo. Integer rutrum turpis ac erat tristique cursus. Sed venenatis urna vel tempus venenatis. Nam eu rhoncus eros, et condimentum elit. Quisque risus turpis, aliquam eget euismod id, gravida in odio. Nunc elementum nibh risus, ut faucibus mauris molestie eu.

Vivamus quis nunc nec nisl vulputate fringilla. Duis tempus libero ac justo laoreet tincidunt. Fusce sagittis gravida magna, pharetra venenatis mauris semper at. Nullam eleifend felis a elementum sagittis. In vel turpis eu metus euismod tempus eget sit amet tortor. Donec eu rhoncus libero, quis iaculis lectus. Aliquam erat volutpat. Proin id ullamcorper tortor. Fusce vestibulum a enim non volutpat. Nam ut interdum nulla. Proin lacinia felis malesuada arcu aliquet fringilla. Aliquam condimentum, tellus eget maximus porttitor, quam sem luctus massa, eu fermentum arcu diam ac massa. Praesent ut quam id leo molestie rhoncus. Praesent nec odio eget turpis bibendum eleifend non sit amet mi. Curabitur placerat finibus velit, eu ultricies risus imperdiet ut. Suspendisse lorem orci, luctus porta eros a, commodo maximus nisi.

Nunc et dolor diam. Phasellus eu justo vitae diam vehicula tristique. Vestibulum vulputate cursus turpis nec commodo. Etiam elementum sit amet erat et pellentesque. In eu augue sed tortor mollis tincidunt. Mauris eros dui, sagittis vestibulum vestibulum vitae, molestie a velit. Donec non felis ut velit aliquam convallis sit amet sit amet velit. Aliquam vulputate, elit in lacinia lacinia, odio lacus consectetur quam, sit amet facilisis mi justo id magna. Curabitur aliquet pulvinar eros. Cras metus enim, tristique ut magna a, interdum egestas nibh. Aenean lorem odio, varius a sollicitudin non, cursus a odio. Vestibulum ante ipsum primis in faucibus orci luctus et ultrices posuere cubilia Curae;

Morbi bibendum est aliquam, hendrerit dolor ac, pretium sem. Nunc molestie, dui in euismod finibus, nunc enim viverra enim, eu mattis mi metus id libero. Cras sed accumsan justo, ut volutpat ipsum. Nam faucibus auctor molestie. Morbi sit amet eros a justo pretium aliquet. Maecenas tempor risus sit amet tincidunt tincidunt. Curabitur dapibus gravida gravida. Vivamus porta ullamcorper nisi eu molestie. Ut pretium nisl eu facilisis tempor. Nulla rutrum tincidunt justo, id placerat lacus laoreet et. Sed cursus lobortis vehicula. Donec sed tortor et est cursus pellentesque sit amet sed velit. Proin efficitur posuere felis, porta auctor nunc. Etiam non porta risus. Pellentesque lacinia eros at ante iaculis, sed aliquet ipsum volutpat. Suspendisse potenti.

Ut ultrices lectus sed sagittis varius. Nulla facilisi. Nullam tortor sem, placerat nec condimentum eu, tristique eget ex. Nullam pretium tellus ut nibh accumsan elementum. Aliquam posuere gravida tellus, id imperdiet nulla rutrum imperdiet. Nulla pretium ullamcorper quam, non iaculis orci consectetur eget. Curabitur non laoreet nisl. Maecenas lacinia, lorem vel tincidunt cursus, odio lorem aliquet est, gravida auctor arcu urna id enim. Morbi accumsan bibendum ipsum, ut maximus dui placerat vitae. Nullam pretium ac tortor nec venenatis. Nunc non aliquet neque.

\section{Some basics}

Let us start with \textbf{some bold font} and then \textit{some forced italic} and contrast that with \emph{emphasise} (which may not be italic in some \LaTeX{} packages). Why not include a footnote that appears at the bottom of the page\footnote{My footnote goes blah blah blah! \dots}.

\section[Short title]{Reasonably long section title}

For this section notice how the section title in the index has been replaced by an alternative.

\section{Examples of lists}

How do you put a list in a \LaTeX{} document? Use \emph{enumerate}, \emph{itemize}, or \emph{description}.

\subsection*{Enumeration}
\begin{enumerate}
\item The first topic is dull
\item The second topic is duller
\begin{enumerate}
\item The first subtopic is silly
\item The second subtopic is stupid
\end{enumerate}
\item The third topic is the dullest
\end{enumerate}

\subsection*{Itemize}
\begin{itemize}
\item The first topic is dull
\item The second topic is duller
\begin{itemize}
\item The first subtopic is silly
\item The second subtopic is stupid
\end{itemize}
\item The third topic is the dullest
\end{itemize}

\subsection*{Description}
\begin{description}
\item[The first topic] is dull
\item[The second topic] is duller
\begin{description}
\item[The first subtopic] is silly
\item[The second subtopic] is stupid
\end{description}
\item[The third topic] is the dullest
\end{description}

\subsection*{Advanced enumeration}

The \LaTeX{} package \textbf{enumitem} must be left uncommented in the \textbf{preamble.tex} for these lists to display correctly. Firstly different alphas for first ordering list items and space between items removed.

\begin{enumerate}[label=\emph{\alph*}), noitemsep]
\item The first topic is dull
\item The second topic is duller
\begin{enumerate}[noitemsep]
\item The first subtopic is silly
\item The second subtopic is stupid
\end{enumerate}
\item The third topic is the dullest
\end{enumerate}

And now a legal list.

\subsubsection{Legal numbering}

\newlist{legal}{enumerate}{10}
\setlist[legal]{label*=\arabic*.}

\begin{legal}
\item The first topic is dull
\item The second topic is duller
\begin{legal}
\item The first subtopic is silly
\item The second subtopic is stupid
\end{legal}
\item The third topic is the dullest
\end{legal}

For other options to change the formatting of lists see the documentation on the \textbf{enumitem} package on CTAN.

\section{Equations and a title with math \texorpdfstring{$\sigma$}{[sigma]}} 

%In-text equation
The most famous equation in the world (as an in-line equation): $E^2 = (m_0c^2)^2 + (pc)^2$, which is known as the \textbf{energy-mass-momentum} relation. Now for another equation.

%An equation
\begin{align}
CIF: \hspace*{5mm}F_0^j(a) = \frac{1}{2\pi \iota} \oint_{\gamma} \frac{F_0^j(z)}{z - a} dz
\end{align}

Followed by another in-line random mapping equation $f:\mathcal{X}\mapsto\mathcal{Z}$ to end the equations bit.

%Defining different nomenclature's for the nomenclature list
\nomenclature[z-cif]{$CIF$}{Cauchy's Integral Formula}                  % first letter Z is for Acronyms 
\nomenclature[a-F]{$F$}{complex function}                               % first letter A is for Roman symbols
\nomenclature[g-p]{$\pi$}{ $\simeq 3.14\ldots$}                         % first letter G is for Greek Symbols
\nomenclature[g-i]{$\iota$}{unit imaginary number $\sqrt{-1}$}          % first letter G is for Greek Symbols
\nomenclature[g-g]{$\gamma$}{a simply closed curve on a complex plane}  % first letter G is for Greek Symbols
\nomenclature[x-i]{$\oint_\gamma$}{integration around a curve $\gamma$}	% first letter X is for Other Symbols
\nomenclature[r-j]{$j$}{superscript index}                              % first letter R is for superscripts
\nomenclature[s-0]{$0$}{subscript index}                                % first letter S is for subscripts

% Comment out the following, when you don't have siunitx package loaded.
\section{SI units example}
The SI Units for dynamic viscosity is \si{\newton\second\per\metre\squared}. The SI Unit for electrical resistance is \si{\ohm}.

\section{An image}

Now I'm going to randomly include a picture, Figure~\ref{fig:placeholder}.

\begin{figure}[htbp!] 
\centering    
\includegraphics[width=1.0\textwidth]{Placeholder}
\caption[Placeholder]{This is just a long figure caption for the Public Domain placeholder graphic from openclipart.org}
\label{fig:placeholder}
\end{figure}

% Flush floats and provide a new page
\clearpage

\section{Source code or text file listing}

How about a program listing using \emph{verbatim}.

\begin{verbatim} 
#include <stdio.h>

main( )
{
    printf("hello, world\n");
}
\end{verbatim}

\subsection{Listings example}

Alternatively the HTML code below uses the \emph{Listings} package, good for source code and other text listings, see the documentation for Listings on CTAN, or \url{http://texdoc.net/texmf-dist/doc/latex/listings/listings.pdf} to learn about Listings.

\begin{lstlisting}[language=HTML,caption={Hello World in HTML},label={lst:helloworld},float=ht]
<!DOCTYPE html>
<html>
    <head>
        <title>Basic Web Page</title>
    </head>
    <body>
Hello World!
    </body>
</html>
\end{lstlisting}

\subsection{Pseudo code example}

What about some pseudo code using the \emph{algorithm2e} package.

\begin{algorithm}[H]
 \SetAlgoLined
 \KwData{this text}
 \KwResult{how to write algorithm with \LaTeX2e }
 initialization\;
 \While{not at end of this document}{
  read current\;
  \eIf{understand}{
    go to next section\;
    current section becomes this one\;
    }{
    go back to the beginning of current section\;
    }
  }
\caption{How to write algorithms}
\end{algorithm}

See \url{http://tug.ctan.org/macros/latex/contrib/algorithm2e/doc/algorithm2e.pdf} for details and options.

% This section will not be listed in the table of contents
\tochide\section{Hidden section}

(This section will not be listed in the table of contents.)

Lorem ipsum dolor sit amet, consectetur adipiscing elit. In magna nisi, aliquam id blandit id, congue ac est. Fusce porta consequat leo. Proin feugiat at felis vel consectetur. Ut tempus ipsum sit amet congue posuere. Nulla varius rutrum quam. Donec sed purus luctus, faucibus velit id, ultrices sapien. Cras diam purus, tincidunt eget tristique ut, egestas quis nulla. Curabitur vel iaculis lectus. Nunc nulla urna, ultrices et eleifend in, accumsan ut erat. In ut ante leo. Aenean a lacinia nisl, sit amet ullamcorper dolor. Maecenas blandit, tortor ut scelerisque congue, velit diam volutpat metus, sed vestibulum eros justo ut nulla. Etiam nec ipsum non enim luctus porta in in massa. Cras arcu urna, malesuada ut tellus ut, pellentesque mollis risus.Morbi vel tortor imperdiet arcu auctor mattis sit amet eu nisi. Nulla gravida urna vel nisl egestas varius. Aliquam posuere ante quis malesuada dignissim. Mauris ultrices tristique eros, a dignissim nisl iaculis nec. Praesent dapibus tincidunt mauris nec tempor. Curabitur et consequat nisi. Quisque viverra egestas risus, ut sodales enim blandit at. Mauris quis odio nulla. Cras euismod turpis magna, in facilisis diam congue non. Mauris faucibus nisl a orci dictum, et tempus mi cursus.

\section{Referencing Resources}

Humanity advances by building upon previously generated knowledge. This is demonstrated in academia by referencing prior knowledge. Acknowledging the achievements of people, organisations, and our forebears gives credit to past endeavours and is a requirement for honest academic work. The Coventry University preferred referencing style follows the \emph{Publication Manual of the \textbf{A}merican \textbf{P}sychological \textbf{A}ssociation \textbf{7}th Edition}~\parencite{APA72019}, abbreviated to APA7. Previous work by a person or organisation is acknowledged by providing their name and the year the work was published in brackets, as shown in the prior sentence. Alternatively, directly discuss the author(s), as with ~\textcite{Rea85,Ancey1996} for example (with the year in brackets), or just write some text and provide references~\parencite{AAB95,Con90,LM65} as examples. Beyond that, there are many nuances in referencing to cover multiple types of works and details, see the Coventry University information on referencing on their website: \url{https://libguides.coventry.ac.uk/referencing}

\section{Other options}

Read the comments in the \textit{preamble.tex} file to see other features, for example adding to-do notes, with or without text highlighting.

%Now switch to landscape and add three images
\begin{landscape}

% Notice how a section format can be applied but the asterisk is used to not list it in the contents
\section*{Subplots}
I can cite Bicycle for our Minds (see Fig.~\ref{fig:Bicycle-for-our-Minds}) and others, e.g. a Placeholder graphic (Fig.~\ref{fig:aPlaceholder}) or I can cite the whole figure as Fig.~\ref{fig:myGraphics}

\begin{figure}
  \centering
  \begin{subfigure}[b]{0.3\textwidth}
    \includegraphics[width=\textwidth]{Cup-of-Coffee}
    \caption{Coffee is always needed!}
    \label{fig:Cup-of-Coffee}   
  \end{subfigure}             
  \begin{subfigure}[b]{0.3\textwidth}
    \includegraphics[width=\textwidth]{Bicycle-for-our-Minds}
    \caption{Bicycle for our Minds}
    \label{fig:Bicycle-for-our-Minds}
  \end{subfigure}             
  \begin{subfigure}[b]{0.3\textwidth}
    \includegraphics[width=\textwidth]{Placeholder}
    \caption{Placeholder}
    \label{fig:aPlaceholder}
  \end{subfigure}
  \caption{My graphics}
  \label{fig:myGraphics}
\end{figure}

\end{landscape}
